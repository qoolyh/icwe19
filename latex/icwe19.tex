% This is samplepaper.tex, a sample chapter demonstrating the
% LLNCS macro package for Springer Computer Science proceedings;
% Version 2.20 of 2017/10/04
%
\documentclass[runningheads]{llncs}
%
\usepackage{amsmath}
\usepackage{graphicx}
\usepackage{subfig}
% Used for displaying a sample figure. If possible, figure files should
% be included in EPS format.
%
% If you use the hyperref package, please uncomment the following line
% to display URLs in blue roman font according to Springer's eBook style:
% \renewcommand\UrlFont{\color{blue}\rmfamily}

\begin{document}
%
\title{Test-driven Feature Extraction of Web Components}
%
%\titlerunning{Abbreviated paper title}
% If the paper title is too long for the running head, you can set
% an abbreviated paper title here
%
\author{Yonghao Long\inst{1} \and
Yancheng Chen\inst{1} \and
Xiangping Chen\inst{2}}
%
\authorrunning{Long et al.}
% First names are abbreviated in the running head.
% If there are more than two authors, 'et al.' is used.
%
\institute{School of Data and Computer Science, Sun Yat-sen University, Guangzhou, China \and
Institute of Advanced Technology, Sun Yat-sen University, Guangzhou, China}
%
\maketitle              % typeset the header of the contribution
%
\begin{abstract}
The abstract should briefly summarize the contents of the paper in
150--250 words.

\keywords{First keyword  \and Second keyword \and Another keyword.}
\end{abstract}
%
%
%
\input{tex/1_intro}

\section{Related Work}
\subsection{JavaScript Analysis}
\subsection{Web Reuse}
\subsection{Feature Location}

\section{Problem Formulation} 
This section defines the feature extraction problem; and some terminologies for formulating the problem.

\emph{Feature}
A feature represents a code snippet $CS$ in a web component that contains a behavior that.
A feature satisfies the user

\emph{Feature extraction}

Test cases


\section{Hierarchical Genetic Algorithm} 
In this work, we define the \emph{feature} of a web component as a code snippet that implements a specified behavior that the users interest. 
Our method uses test-cases to specify what behaviors the user like, that is, our method will find a minimum size of code snippet which can pass all of the user-specified test-cases. 
We propose a hierarchical genetic algorithm to achieve the goal. 

\subsection{Problem Formulation} 
\begin{definition}
Given a set of test-cases $T$, for a code snippet $CS$ from a web component $C$, we say the $CS$ is the desired features which satisfies:
\end{definition}


\input{tex/5_exp}
\input{tex/6_conclusion}

\bibliographystyle{splncs04}
% \bibliography{mybibliography}
\end{document}
